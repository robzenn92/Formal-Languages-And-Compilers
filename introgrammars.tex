\chapter{Grammatiche e linguaggi}

\section{What is a Grammar?}

A grammar $\mathcal{G}$ is a set. In particular, $\mathcal{G}$ is defined as follows
\begin{equation}
\mathcal{G} = (\mathcal{V}, \mathcal{T}, \mathcal{S}, \mathcal{P})
\end{equation}
where:
\begin{itemize}
\item $\mathcal{V}$ is the non empty set of symbols (terminal and non terminal)
\item $\mathcal{T} \subset \mathcal{V}$ is the set of terminal symbols
\item $\mathcal{S} \in \mathcal{V} \setminus \mathcal{T}$ is the starting symbol (non terminal)
\item $\mathcal{P}$ is the set of productions of the form $\alpha \rightarrow \beta$, $\alpha \in \mathcal{V}^+$, $\beta \in \mathcal{V}^*$.
\end{itemize}

A string $\gamma$ immidiately derives from $\mu$ in $\mathcal{G} = (\mathcal{V}, \mathcal{T}, \mathcal{S}, \mathcal{P})$ iff
\begin{equation}
\mu = \sigma \alpha \tau, \alpha \rightarrow \beta \in \mathcal{P}, \gamma = \sigma \beta \tau \text{ with } \sigma,\tau \in \mathcal{V}^*
\end{equation}
denoted by $\mu \xrightarrow[\mathcal{G}]{}^* \gamma$.

Given $\mathcal{G} = (\mathcal{V}, \mathcal{T}, \mathcal{S}, \mathcal{P})$, then the language generated by $\mathcal{G}$, written $\mathcal{L(G)}$, is defined by
\[
\set{ w | w \in \mathcal{T}^*, \mathcal{S} \xrightarrow[\mathcal{G}]{}^* w }
\]

\section{Dervation tree for a Grammar}
A derivation tree for a Grammar G is a labelled tree wich that:
\begin{enumerate}
\item the label of the root is S
\item a node labelled by $X \in \mathcal{V}$ has ordered immediate descendants \\ $X_1, X_2, \dots X_n \text{ iff } X \rightarrow X_1 \dots X_n \in P$
\item the border $X$ of the tree is a word in $\mathcal{L\set{G}}$ iff $X \in T^*$.
\end{enumerate}

Examples:
\[
\mathcal{G}: S \rightarrow aSb | ab

\mathcal{L(G)} = \set{ a^nb^n | n > 0 } 
\]

\section{Regular Grammars}
Tutti i linguaggi regolari possono essere scritti sottoforma di regular expression oppure sottoforma di automi a stati finiti.

\subsection{Regular Languages}
I linguaggi generati da grammatiche regolari sono detti Regular Languages.
Tutti i linguaggi regolari possono essere scritti sottoforma di regular expression oppure sottoforma di automi a stati finiti.
Let $L_1$ and $L_2$ be Regular Languages.
Regular languages are closed only under:
\begin{itemize}
\item Union: $L_1 \cup L_2 \rightarrow L_{new}$ Regular Language
\item Intersection: $L_1 \cap L_2 \rightarrow L_{new}$ Regular Language
\item Concatenation: $L_1 \cdot L_2 \rightarrow L_{new}$ Regular Language
\item Complementation: $\bar{L_1} \rightarrow L_{new}$ Regular Language
\item Klenee star operator: $L^*$
\end{itemize}
% ---------------------------------------------
\section{Context Free Grammars}
Le grammatiche Context Free sono grammatiche definite dalla quintupla \newline{...}\newline e anche

\subsection{Context Free Languages}
I linguaggi genereati da questo tipo di grammatiche vengono detti, ovviamente, Context Free Languages (CFL).
Let $L_1$ and $L_2$ be Context Free Languages.
Context-Free languages are closed only under:
\begin{itemize}
\item Union: $L_1 \cup L_2 \rightarrow L_3$ Context Free Language
\item Concatenation: $L_1 \cdot L_2 \rightarrow L_{new}$ Regular Language
\item Klenee star operator: $L^*$
\end{itemize}

\section{Regular Languages Vs Context Free Languages}

Come scritto sopra entrambe le grammatiche generano due tipi diversi di linguaggi. Essi sono legati dal fatto che i Regular Languages sono un sottoinsieme dei Context Free Languages.