\chapter{Grammars and languages}

\section{What is a Grammar?}

A grammar $\mathcal{G}$ is a set of rules. In particular, $\mathcal{G}$ is a four tuple defined as follows:
\begin{equation}
\mathcal{G} = (\mathcal{V}, \mathcal{T}, \mathcal{S}, \mathcal{P})
\end{equation}
where:
\begin{itemize}
\item $\mathcal{V}$ is the non empty set of symbols (terminal and non terminal)
\item $\mathcal{T} \subset \mathcal{V}$ is the set of terminal symbols
\item $\mathcal{S} \in \mathcal{V} \setminus \mathcal{T}$ is the starting symbol (non terminal)
\item $\mathcal{P}$ is the set of productions of the form $\alpha \rightarrow \beta$, $\alpha \in \mathcal{V}^+$, $\beta \in \mathcal{V}^*$.
\end{itemize}
\noindent
A string $\gamma$ immidiately derives from $\mu$ in $\mathcal{G} = (\mathcal{V}, \mathcal{T}, \mathcal{S}, \mathcal{P})$ iff
\begin{equation}
\mu = \sigma \alpha \tau, \alpha \rightarrow \beta \in \mathcal{P}, \gamma = \sigma \beta \tau \text{ with } \sigma,\tau \in \mathcal{V}^*
\end{equation}
denoted by $\mu \xrightarrow[\mathcal{G}]{}^* \gamma$.

\noindent
One way to denote a grammar $\mathcal{G}$ is the \textit{Backus-Naur Form} simply called \textit{BNF}.




% inserire esempi di grammatiche...

\section{What is a Language?}
Given $\mathcal{G} = (\mathcal{V}, \mathcal{T}, \mathcal{S}, \mathcal{P})$, then the language generated by $\mathcal{G}$, written $\mathcal{L(G)}$, is defined by
\[
\mathcal{L(G)} = \set{ w | w \in \mathcal{T}^*, \mathcal{S} \xrightarrow[\mathcal{G}]{}^* w }
\]

% inserire esempi di linguaggi...

\subsection{Observation}

\begin{enumerate}
\item Dato un linguaggio $\mathcal{L}$, possono esistere diverse grammatiche $\mathcal{G_j}$ con $j>0$ tali che $\mathcal{L = L(G_j)}$;
\item In generale, dato un linguaggio $\mathcal{L}$ non esiste nessun algoritmo per dimostrare che $\mathcal{L = L(G)}$;
\item In molti casi la 2 può essere verificata dimostrando per induzione che $\mathcal{L} \subseteq \mathcal{L(G)}$ e $\mathcal{L(G)} \subseteq \mathcal{L}$.
\end{enumerate}

\section{Derivation tree for a Grammar}
A derivation tree for a Grammar $\mathcal{G}$ is a labelled tree wich that:
\begin{enumerate}
\item the label of the root is S
\item a node labelled by $X \in \mathcal{V}$ has ordered immediate descendants \\ $X_1, X_2, \dots X_n \text{ iff } X \rightarrow X_1 \dots X_n \in P$
\item the border $X$ of the tree is a word in $\mathcal{L\set{G}}$ iff $X \in T^*$.
\end{enumerate}

\noindent
Example:
\[
\begin{aligned}
\mathcal{G}: S \rightarrow aSb | ab
\quad
\mathcal{L(G)} = \set{ a^nb^n | n > 0 }
\end{aligned}
\]
Three different derivation trees for the grammar $\mathcal{G}$ could be:
\begin{center}
\synttree	[S 	[a]
				[b]
			]
\quad
\synttree	[S 	[a]
		    	[S 	[a]
		       		[b]
				]
				[b]
			]
\quad
\synttree	[S 	[a]
		    	[S 	[a]
		    		[S 	[a]
		    			[b]
		    		]
		       		[b]
				]
				[b]
			]
\end{center}
\noindent
The only way to read these tree is from the left to the right, so we can recognize the words 'ab', 'aabb', 'aaabbb' of the language $\mathcal{L(G)}$.

\noindent
Exercise: Provide a Grammar $\mathcal{G}$ that can recognize the language $\mathcal{L(G)} = \set{ ww^R | w\text{ is }a^*b^*}$.
\noindent
Solution: the language $\mathcal{L(G)}$ generates palindrome words so it can generate 'aa', 'aba', 'abba', 'abaaba', \dots.

\noindent
Possible solutions are:
\[
\begin{aligned}
\mathcal{G_\text{1}}: &S \rightarrow CD\\
&C \rightarrow aCa | bCb | \epsilon\\
&D \rightarrow \epsilon
\end{aligned}
\quad
\begin{aligned}
\mathcal{G_\text{2}}: &S \rightarrow C\\
&C \rightarrow aCa | bCb | \epsilon
\end{aligned}
\quad
\begin{aligned}
\mathcal{G_\text{3}}: &S \rightarrow aSa | bSb | \epsilon\\
\end{aligned}
\]

\section{Regular Grammars}
Tutti i linguaggi regolari possono essere scritti sottoforma di regular expression oppure sottoforma di automi a stati finiti.

\subsection{Regular Languages}
I linguaggi generati da grammatiche regolari sono detti Regular Languages [RL].
Tutti i linguaggi regolari possono essere scritti sottoforma di regular expression oppure sottoforma di automi a stati finiti.
Let $\mathcal{L_1}$ and $\mathcal{L_2}$ be Regular Languages.
Regular languages are closed only under:
\begin{itemize}
\item Union: $L_1 \cup L_2 \rightarrow L_{new}$ Regular Language
\item Intersection: $L_1 \cap L_2 \rightarrow L_{new}$ Regular Language
\item Concatenation: $L_1 \cdot L_2 \rightarrow L_{new}$ Regular Language
\item Complementation: $\bar{L_1} \rightarrow L_{new}$ Regular Language
\item Klenee star operator: $L^*$
\end{itemize}
% ---------------------------------------------
\section{Context Free Grammars}
Le grammatiche Context Free sono grammatiche definite dalla quintupla \newline{...}\newline e anche

\subsection{Context Free Languages}
I linguaggi genereati da questo tipo di grammatiche vengono detti, ovviamente, Context Free Languages (CFL).
Let $L_1$ and $L_2$ be Context Free Languages.
Context-Free languages are closed only under:
\begin{itemize}
\item Union: $L_1 \cup L_2 \rightarrow L_3$ Context Free Language
\item Concatenation: $L_1 \cdot L_2 \rightarrow L_{new}$ Regular Language
\item Klenee star operator: $L^*$
\end{itemize}

\section{Regular Languages Vs Context Free Languages}

Come scritto sopra entrambe le grammatiche generano due tipi diversi di linguaggi. Essi sono legati dal fatto che i Regular Languages sono un sottoinsieme dei Context Free Languages.