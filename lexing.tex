% Le regular expression sono delle espressioni che permettono di genereare un linguaggio. In particolare sia r una regolar expression definita come segue:
% \begin{equation}
% r = a^*b^*a|b^+abb
% \end{equation}
% Nella definizione delle regular expression occorre prestare attenzione ad alcune semplici regole. Siano a,b lettere qualsiasi di un Linguaggio L allora in termini di regular expression:

% \begin{itemize}
% \item $a^+$ equivale a dire <<una o più ripetizioni di a>>
% \item $a^*$ equivale a dire <<zero o più ripetizioni di a>>, vale $a^+ = aa^*$
% \item $a|b$ equivale a dire <<a oppure b>>, in un passo posso prendere una sola a oppure una sola b
% \item $ab$ equivale alla concatenazione di una a seguita da una b.
% \end{itemize}

% Di fronte ad una regular expression valgono le seguenti regole di priorità:

% \begin{enumerate}
% \item $*$ Klenee star operator, priorità assoluta.
% \item $|$ Pipe
% \item $.$ Concatenazione
% \item $($ o $)$ Parentesi
% \end{enumerate}

% Quindi nel caso della (1.1) si ha:

% \begin{enumerate}
% \item $a^*$ sempre per prima
% \item $b^*$ subito dopo $a^*$
% \item $a$ oppure $b^+$
% \item $abb$ come concatenzazione finale
% \end{enumerate}

\chapter{Lexing}

\section{Regular expressions}

Regular expressions are sequences of characters allow to generate a Language $\mathcal{L(G)}$. A regular expression could be defined as follows:
\begin{equation}
r = a^*b^*a|b^+abb
\end{equation}
Any regular expression r denotes a language $\mathcal{L}\set{r}$.
For each a $\in\mathcal{A}$ if r=a then $\mathcal{L}\set{\text{r}}=\set{a}$, if r=$\epsilon$ then $\mathcal{L}\set{r}=\set{\epsilon}$. In a regular expression r we could find:

\begin{itemize}
\item \textit{Alternation} with the pipe symbol. \\ If r=$r_1$|$r_2$ then $\mathcal{L}\set{r} = \mathcal{L}\set{r_1} \cup \mathcal{L}\set{r_2}$.
\item \textit{Concatenation} with the dot symbol (ommitted). \\ If r=$r_1r_2$ then $\mathcal{L}\set{w_1w_2 | w_1 \in \mathcal{L}\set{r_1}, w_2 \in \mathcal{L}\set{r_2}}$.
\item \textit{Kleenestar repetition} with the star symbol. \\ If r=$r_1^*$ then $\mathcal{L}\set{r}=\set{\epsilon} \cup \set{ww^k | w \in \mathcal{L}\set{r}, k \geq 0}$.
\end{itemize}
All this possibilities has a precedence list:
\begin{enumerate}
\item Kleenestar like $r^+$ has the highest precedence.
\item Concatenation like $r_1r_2$ has the second highest precedence.
\item Alternation like $r_1$|$r_2$ has the lowest precedence and it is also left associative.
\end{enumerate}
For example:
\begin{equation}
r = a|b^*c = ((a)|(b^*))c
\end{equation}

\section{Automata}

Automata are graphs so they consists of states (represented by circles), and transitions (represented by arrows).
We can distinguish two kinds of automata:
\begin{itemize}
\item \textit{NFA} stands for Non Deterministic finite automata.
\item \textit{DFA} stands for Deterministic finite automata.
\end{itemize}

\subsection{NFA}

An NFA is a finite state automaton defined as follows:
\begin{equation}
(\mathcal{S} , \mathcal{A} , move , s_0 , \mathcal{F})
\end{equation}
where:
\begin{itemize}
\item $\mathcal{S}$ is the set of states;
\item $\mathcal{A}$
\item $move$ is a transition function defined as follows: $\mathcal{S} \times (\mathcal{A}\cup \set{\epsilon} ) \rightarrow 2^s$;
\item $s_0$ is a particolar state callsed "initial state", $s_0 \in \mathcal{S}$
\item $\mathcal{F}$ is the set of final states, $\mathcal{F} \subset \mathcal{S}$
\end{itemize}
\newline
An NFA $\mathcal{N}$ accepts a string \textit{w} iff \textit{there exist at least a path} in $\mathcal{N}$ from $s_0$ to $s_j \in \mathcal{F}$ for some $s_j$. The language recognized by $\mathcal{N}$, $\mathcal{L\set{N}}$, is the set of strings accepted by $\mathcal{N}$.

\subsection{DFA}

A DFA has the same definition as an NFA:
\begin{equation}
(\mathcal{S} , \mathcal{A} , move , s_0 , \mathcal{F})
\end{equation}

However, a DFA is a nibcase of an NFA in which there are no $\epsilon-move$.
For each state $s \in \mathcal{S}$ and for each $a \in \mathcal{A}$ there is \textit{exactly one} edge outgoing s and labelled by "a". In other hand, move function is \textit{total} and it is defined as:
\begin{equation}
move: \mathcal{S} \times a \rightarrow \mathcal{S}
\end{equation}











