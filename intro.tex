\chapter{Introduction to the course}

A \textit{compiler} is a specific software that takes a source program and transform it into executable code. There are 2 different compilation phases:
\begin{itemize}
\item \textit{Front-End} part: takes source code and transform it into intermediate code (indipendent of the hardware you have)
\item \textit{Back-End} part: takes the intermediate code generated before and transform it into machine code (it depends of the hardware you have).
\end{itemize}

\section{Front-End}

The Front-End part of a compiler is composed by the \textit{lexical analysis}. It produces in output a stream of tokens able to recognize keywords as atomic units, i.e. identifiers.
After the lexical analisys there is the \textit{syntax analysis} that verify the roles of the grammar that define a particular language.
Then there is the \textit{static analysis} where the type checking control is computed. Finally there is an \textit{optimization process} and the generation of the \textit{intermediate code}.

% analisi lessicale produce in output uno stream di tokens in grado di riconoscere ad esempio keyword del linguaggio, identificatori, ecc. come unità atomiche.

% analisi sintattica: verifica che siano rispettate le regole della grammatica che definisce il lunguaggio. (nota come parsing)
% costruzione di un albero di derivazione per tenere conto delle precedenze degli operatori.

% analisi statica: type checking (semantica statica)

% ottimizzazione

% generazione del codice intermedio

% ultime 2 = semantica dinamica

\section{Back-End}

The back-End part of a compiler is basically composed by the generation of the code for a particular architecture. Also an optimization process for the specific hardware could be done in this part.
\newline
For example, given an (high level) expression like this:
\[
position = initial + rate * 60;
\]
we obtain an intermediate code common for all architectures and then the machine code optimized for a particular hardware.
\[
\left.
\begin{aligned}
&temp1	:=	intoreal(60);\\
&temp2	:=	id_3 + temp1;\\
&temp3	:=	id_2 + temp2;\\
&id_1	:=	temp3;
\end{aligned}
\quad
\xrightarrow{\textit{optimization}}
\quad
\begin{aligned}
&MOVF\;id_3, R2;\\
&MULT\;\#60.0, R2;\\
&MOVF\;id_2, R1;\\
&ADDF\;R2, R1;\\
&MOVF\;R1, id_1;
\end{aligned}
\right.
\]